\section{Observational Study}
\subsection{Covariate Balance Plot}
\begin{figure}
\centering
\includegraphics[width=0.95\textwidth]{balancePlot.pdf}
\caption{Standardized differences on baseline covariates between
  students in the treatment group with $H=0$ and $H=1$, before and
  after propensity score matching. The dotted vertical lines are at
  $\pm$0.25 and $\pm$0.05, corresponding to What Works Clearinghouse
  standards.}
\label{fig:balance}
\end{figure}

Figure \ref{fig:balance} shows covariate balance before and after the
match.
The red squares in the figure represent covariate standardized differences
between the two groups before adjustment; apparently, $H=1$ and $H=0$
students varied on a number of characteristics.
$H=1$ students tended to have lower pretest scores, were less likely
to be fluent in English, gifted, white, or Asian, and in ninth grade,
but more likely to need special education.

\subsection{Continuous Measure of Hint Use}

As an alternative to treating $H$ as binary, we may model hint requests continuously, via
the IRT parameter $\eta$.
Doing so effectively assumes a stronger version of matched
ignorability: within matched sets potential outcomes $Y(\eta)$ are
independent of $\eta$.
Adding covariates to the regression weakens this assumption, by using
the regression model to adjust for observed confounding between $\eta$
and $Y(\eta)$ left over after the match.

We hypothesized that the effect of variance in $\eta$ may not be
monotonic: that there is an optimal proclivity to request a hint that
is greater than zero.
Students asking for too many  hints may be wheel spinning
\citep[c.f.][]{wheelSpinning} or not putting in sufficient effort to
learn the material, while students who request too few hints are not
taking advantage of a useful learning resource.
To estimate the effect of $\eta$ on posttest
scores---i.e. $Y(\eta=r_1)-Y(\eta=r_0)$ for $r_1$ and $r_2$ two
hypothetical values of $\eta$---we regressed $Y$ on matched set
indicators, the full set of covariates, and penalized splines for
pretest and $\eta$, using the \texttt{gam} function of the
\texttt{mgcv} package in \texttt{R} \citep{mgcv}.

Figure \ref{fig:gam} plots estimates of $\eta$ against adjusted
posttest scores, with the estimated effects of all the regressors
other than $\eta$ subtracted out.
The estimated effect of $\eta$ is plotted as a line, with the dotted
lines giving an approximate 95\% confidence region.
Our hypothesis of a non-monotonic $\eta$ effect was not borne out:
the fitted line is monotonically negative throughout.

\begin{figure}
\centering
\includegraphics[width=0.95\textwidth]{continuousMatched.pdf}
\caption{Results from a model estimating the effect of continuous hint
  parameter $\eta$ on posttest scores. Each point on the graph plots a
  the posterior mode of $\eta_i$ for subject $i$ in the treatment
  group against $Y_{adj}$, $i$'s posttest score minus the model's
  predictions based on all regressors other than $\eta$. The slope of the
  regression line is the estimated effect of local changes in $\eta$
  on $Y$.}
\label{fig:gam}
\end{figure}
